\documentclass[a4paper,11pt]{jsarticle}
\usepackage{bm}
\usepackage[dvipdfmx]{graphicx}
\usepackage{ascmac}
\usepackage{otf}
\pagestyle{plain}
%\setcounter{page}{}
\usepackage[dvipdfmx]{hyperref}
\usepackage{pxjahyper}

\begin{document}
\begin{center}
{\Large 令和二年度 卒業研究}\\
{\HUGE ポリエチレン標的を用いた\\陽子散乱実験}\\
\vspace*{8cm}
{\LARGE 北野亮輔}\\
{\LARGE 大谷萌}\\
{\LARGE 金\UTF{FA11}奎}\\
{\LARGE 木村将}\\
\vspace*{0.5cm}
{\LARGE 神戸大学理学部物理学科 粒子物理学研究室}\\
\vspace*{2cm}
{\LARGE 2021年 3月19日}
\end{center}
\thispagestyle{empty}

\newpage
\thispagestyle{empty}
 
\newpage
{\LARGE 概要}\\

Ernest Rutherford(1871)は$\alpha$線や$\beta$線の発見、人工的な核変換の成功、など多くの功績を残し、「原子核物理学の父」と呼ばれている。彼の業績の中で最も広く知られているのがRutherfordの散乱実験における原子核の発見であろう。それまで原子の構造については、J.J.Tomsonによるいわゆる「ブドウパンモデル」などが提唱されていたが、広く受け入れられなかった。Rutherfordらは実験の中で$\alpha$線を金原子に照射すると、その一部が大きな偏向角を持って散乱されることを発見した。この結果から、原子の中心に正電荷を持つ小さな核があり、その周囲を電子が回転運動しているという「Rutherfordの原子模型」を提唱した。

本研究では標的にポリエチレンと金を採用し、神戸大学海事科学部にあるタンデム静電加速器を用いて3MeVの陽子を入射した。各散乱角度における微分散乱断面積を求めることによって理論値との比較を行い、原子核の半径やその大きさによるRutherfordの散乱公式の補正や、同種粒子の散乱を見ることを目指した。


\thispagestyle{empty}

\newpage
\thispagestyle{empty}
\setcounter{tocdepth}{3}
\tableofcontents
\thispagestyle{empty}
\thispagestyle{empty}
\end{document}


