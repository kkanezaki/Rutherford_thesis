\documentclass[a4paper,11pt,dvipdfmx]{jsarticle}

\usepackage{bm}
\usepackage[dvipdfmx]{graphicx}
\usepackage[dvipdfmx]{color}
\usepackage{ascmac}
\usepackage{siunitx}
\usepackage{otf}
\pagestyle{plain}
\usepackage{float}
\usepackage[dvipdfmx]{hyperref}
\usepackage{pxjahyper}
\usepackage{here}
\usepackage{titlesec}
\titleformat*{\section}{\LARGE\bfseries}
\titleformat*{\subsection}{\normalsize\bfseries}
\usepackage{url}
\hypersetup{% hyperrefオプションリスト
setpagesize=false,
 bookmarksnumbered=true,%
 bookmarksopen=true,%
 colorlinks=true,%
 linkcolor=blue,
 citecolor=blue,
}


\begin{document}
\newpage
\phantomsection
{\LARGE 謝辞}\\
\addcontentsline{toc}{section}{謝辞}

本論文は私たちが四年次に行った卒業研究の成果をまとめたものです。研究の計画及び遂行、論文の執筆にあたりご指導及びご協力いただきました多くの方々に対し心より感謝申し上げます。

越智敦彦准教授には本研究の担当教員として大変お世話になりました。普段のミーティングでは先生のお持ちの幅広い知識から様々なアドバイスをいただき、一を聞けば十七ほど知ることができました。海事科学部での本実験では、加速器をはじめとする実験装置の扱いから指導していただき、実験がうまくいかない時は夜遅くまで我々の実験に付き合っていただきました。本当にありがとうございました。また某びっくり系ハンバーグレストランでマーメイドサラダのみをおかずに白米を食されており、食に対しても新たな可能性を探られる姿勢には大変感銘を受けました。

その他の神戸大学粒子物理学研究室の皆様にもお礼申し上げます。藏重久弥教授、竹内康雄教授、山崎祐司教授、身内賢太朗准教授、前田順平講師、鈴木州助教、中野佑樹特命助教、東野聡研究員、吉田和美秘書 には実験のための快適な環境を用意していただいたり、ゼミで指導していただいたり、発表の際にはその都度適切な助言をいただきました。一つ発表すれば二十八ほどご指摘をいただき、おかげさまでこの卒業論文もなんとか形にすることができました。ありがとうございました。また身内准教授をはじめとするダークマターグループの皆様や前田講師には、実験に使用する機器を快く貸していただきました。D2の水越彗太さん、M2の島田拓弥さんにはVMEモジュールを用いたDAQの構築の際、お世話になりました。その他の修士・博士課程の先輩方や、同期の皆さんにも日頃から仲良くさせていただきました。これからの皆さんのご多幸をお祈りいたします。

最後に大学生活を支えていただいた全ての方々と、何より健気に原子核に衝突してくれた陽子ちゃんたちに感謝の意を示し謝辞と致します。

\end{document}