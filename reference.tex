\documentclass[a4paper,11pt,dvipdfmx]{jsarticle}

\usepackage{bm}
\usepackage[dvipdfmx]{graphicx}
\usepackage[dvipdfmx]{color}
\usepackage{ascmac}
\usepackage{siunitx}
\usepackage{otf}
\pagestyle{plain}
\usepackage{float}
\usepackage[dvipdfmx]{hyperref}
\usepackage{pxjahyper}
\usepackage{here}
\usepackage{titlesec}
\titleformat*{\section}{\LARGE\bfseries}
\titleformat*{\subsection}{\normalsize\bfseries}
\usepackage{url}
\hypersetup{% hyperrefオプションリスト
setpagesize=false,
 bookmarksnumbered=true,%
 bookmarksopen=true,%
 colorlinks=true,%
 linkcolor=blue,
 citecolor=blue,
}


\begin{document}

\newpage
\phantomsection
\begin{thebibliography}{99}
\addcontentsline{toc}{section}{参考文献}

  \bibitem{2017} 石飛由介、岡田健、桑野将大、杉本太郎、吉田登志輝 \\
『平成28年度 卒業論文 ラザフォード散乱』, 2017年
  \\
  \bibitem{ion} 工藤博『イオンビーム工学入門:論文を読むための基礎知識』, 2018年
  \\
  \bibitem{kyodai} 加須屋春樹、坂口雄一、鈴木一輝、中脇稔貴、藤井涼平 \\
『2015年前期課題演習A5報告書』, 2015年
  \\
  \bibitem{introto} Alessandro Bettini, \textit{Introduction to Elementary Particle Physics}, \\ 
CAMBRIDGE UNIVERSITY PRESS, 2012
  \\
  \bibitem{eneloss} 岡村勇介『荷電粒子の物質中でのエネルギー損失と飛程』, 2009年
  \\ %北野ここまで、以下大谷
  
    \bibitem{tandemu}[タンデム加速器とは]タンデム加速器・タンデムブースタ\\
  \url{https://ttandem.jaea.go.jp/koumoku-02/kasokuki.html}\\
  
  \bibitem{beam}荷電粒子ビーム実験\\
  \url{http://www2.kobe-u.ac.jp/~taniikea/particlebeam.pdf}\\
  
  \bibitem{kasokukishiyou}
  \url{http://www.research.kobe-u.ac.jp/fmsc-pbe/www/5sdh2/5sdh2.html}\\
  
  \bibitem{syousen}
  神戸商船大学紀要,第ニ類,商船・理工学篇 第46号(神戸商船大学出版 1998年7月出版)\\
  
  \bibitem{pin}フォトダイオード(PD)の構造や原理とは | ファイバーラボ株式\\
  \url{https://www.fiberlabs.co.jp/tech-explan/about-pd/}\\

  %大谷ここまで
  \bibitem{2019} タンデム静電加速器を用いたラザフォード散乱実験, 2020年\\
  \url{https://ppwww.phys.sci.kobe-u.ac.jp/seminar/pdf/Rutherford_2019.pdf}\\
  
  \bibitem{vme}
  \url{http://www-cr.scphys.kyoto-u.ac.jp/research/cangaroo/OLDWEB/design2.html}\\
  
  \bibitem{BBT} SiTCP VME-Master module Mode2 BBT-002-2 取扱説明書\\
  \url{https://www.bbtech.co.jp/download-files/vmegbe/SiTCP-VME-Master(Rev26).pdf}\\
  
  \bibitem{hoshin} 豊伸電子HP 製品情報\\
  \url{https://www.kagaku.com/hoshin/index.html}\\
  
  \bibitem{repic} ハヤシレピックHP 製品情報(RPN-110)\\
  \url{https://www.h-repic.co.jp/products/module/module_nim/rpn_110}\\
  
  \bibitem{braggpeak} Bragg peak - Wikipedia \\
  \url{https://en.wikipedia.org/wiki/Bragg_peak}\\
  
  \bibitem{wave} \url{http://qo.phys.gakushuin.ac.jp/~torii/Dthesis/pdf/付録D.pdf}
\end{thebibliography}

\end{document}