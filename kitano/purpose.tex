\documentclass[a4paper,11pt,dvipdfmx]{jsarticle}

\usepackage{bm}
\usepackage[dvipdfmx]{graphicx}
\usepackage[dvipdfmx]{color}
\usepackage{ascmac}
\usepackage{siunitx}
\usepackage{otf}
\pagestyle{plain}
\usepackage{float}
\usepackage[dvipdfmx]{hyperref}
\usepackage{pxjahyper}
\usepackage{here}
\usepackage{titlesec}
\titleformat*{\section}{\LARGE\bfseries}
\titleformat*{\subsection}{\normalsize\bfseries}
\usepackage{url}
\usepackage[table,xcdraw]{xcolor}
\hypersetup{% hyperrefオプションリスト
setpagesize=false,
 bookmarksnumbered=true,%
 bookmarksopen=true,%
 colorlinks=true,%
 linkcolor=blue,
 citecolor=blue,
}

\begin{document}
\newpage
\section{\LARGE{導入と理論(担当:北野)}}

\subsection{実験の目的と概要}
我々は、神戸大学深江キャンパスにあるタンデム加速器を用いてポリエチレン標的を用いた陽子の散乱実験を行った。本実験の目的は、2つの検出器を用いて陽子-陽子散乱(p-p散乱)の同時計測を行うこと、またポリエチレンに含まれる炭素原子核によって散乱される陽子の微分散乱断面積を測定し、炭素の原子核半径を求めることである。

本論文では本実験の理論、実験装置、データ収集と解析、議論と考察について述べる。

\end{document}